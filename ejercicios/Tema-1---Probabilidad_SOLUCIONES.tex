% Options for packages loaded elsewhere
\PassOptionsToPackage{unicode}{hyperref}
\PassOptionsToPackage{hyphens}{url}
\PassOptionsToPackage{dvipsnames,svgnames*,x11names*}{xcolor}
%
\documentclass[
]{article}
\usepackage{lmodern}
\usepackage{amssymb,amsmath}
\usepackage{ifxetex,ifluatex}
\ifnum 0\ifxetex 1\fi\ifluatex 1\fi=0 % if pdftex
  \usepackage[T1]{fontenc}
  \usepackage[utf8]{inputenc}
  \usepackage{textcomp} % provide euro and other symbols
\else % if luatex or xetex
  \usepackage{unicode-math}
  \defaultfontfeatures{Scale=MatchLowercase}
  \defaultfontfeatures[\rmfamily]{Ligatures=TeX,Scale=1}
\fi
% Use upquote if available, for straight quotes in verbatim environments
\IfFileExists{upquote.sty}{\usepackage{upquote}}{}
\IfFileExists{microtype.sty}{% use microtype if available
  \usepackage[]{microtype}
  \UseMicrotypeSet[protrusion]{basicmath} % disable protrusion for tt fonts
}{}
\makeatletter
\@ifundefined{KOMAClassName}{% if non-KOMA class
  \IfFileExists{parskip.sty}{%
    \usepackage{parskip}
  }{% else
    \setlength{\parindent}{0pt}
    \setlength{\parskip}{6pt plus 2pt minus 1pt}}
}{% if KOMA class
  \KOMAoptions{parskip=half}}
\makeatother
\usepackage{xcolor}
\IfFileExists{xurl.sty}{\usepackage{xurl}}{} % add URL line breaks if available
\IfFileExists{bookmark.sty}{\usepackage{bookmark}}{\usepackage{hyperref}}
\hypersetup{
  pdftitle={SOLUCIONES: Ejercicios Tema 1 - Probabilidad.},
  pdfauthor={Ricardo Alberich, Juan Gabriel Gomila y Arnau Mir},
  colorlinks=true,
  linkcolor=Maroon,
  filecolor=Maroon,
  citecolor=Blue,
  urlcolor=Blue,
  pdfcreator={LaTeX via pandoc}}
\urlstyle{same} % disable monospaced font for URLs
\usepackage[margin=1in]{geometry}
\usepackage{color}
\usepackage{fancyvrb}
\newcommand{\VerbBar}{|}
\newcommand{\VERB}{\Verb[commandchars=\\\{\}]}
\DefineVerbatimEnvironment{Highlighting}{Verbatim}{commandchars=\\\{\}}
% Add ',fontsize=\small' for more characters per line
\usepackage{framed}
\definecolor{shadecolor}{RGB}{248,248,248}
\newenvironment{Shaded}{\begin{snugshade}}{\end{snugshade}}
\newcommand{\AlertTok}[1]{\textcolor[rgb]{0.94,0.16,0.16}{#1}}
\newcommand{\AnnotationTok}[1]{\textcolor[rgb]{0.56,0.35,0.01}{\textbf{\textit{#1}}}}
\newcommand{\AttributeTok}[1]{\textcolor[rgb]{0.77,0.63,0.00}{#1}}
\newcommand{\BaseNTok}[1]{\textcolor[rgb]{0.00,0.00,0.81}{#1}}
\newcommand{\BuiltInTok}[1]{#1}
\newcommand{\CharTok}[1]{\textcolor[rgb]{0.31,0.60,0.02}{#1}}
\newcommand{\CommentTok}[1]{\textcolor[rgb]{0.56,0.35,0.01}{\textit{#1}}}
\newcommand{\CommentVarTok}[1]{\textcolor[rgb]{0.56,0.35,0.01}{\textbf{\textit{#1}}}}
\newcommand{\ConstantTok}[1]{\textcolor[rgb]{0.00,0.00,0.00}{#1}}
\newcommand{\ControlFlowTok}[1]{\textcolor[rgb]{0.13,0.29,0.53}{\textbf{#1}}}
\newcommand{\DataTypeTok}[1]{\textcolor[rgb]{0.13,0.29,0.53}{#1}}
\newcommand{\DecValTok}[1]{\textcolor[rgb]{0.00,0.00,0.81}{#1}}
\newcommand{\DocumentationTok}[1]{\textcolor[rgb]{0.56,0.35,0.01}{\textbf{\textit{#1}}}}
\newcommand{\ErrorTok}[1]{\textcolor[rgb]{0.64,0.00,0.00}{\textbf{#1}}}
\newcommand{\ExtensionTok}[1]{#1}
\newcommand{\FloatTok}[1]{\textcolor[rgb]{0.00,0.00,0.81}{#1}}
\newcommand{\FunctionTok}[1]{\textcolor[rgb]{0.00,0.00,0.00}{#1}}
\newcommand{\ImportTok}[1]{#1}
\newcommand{\InformationTok}[1]{\textcolor[rgb]{0.56,0.35,0.01}{\textbf{\textit{#1}}}}
\newcommand{\KeywordTok}[1]{\textcolor[rgb]{0.13,0.29,0.53}{\textbf{#1}}}
\newcommand{\NormalTok}[1]{#1}
\newcommand{\OperatorTok}[1]{\textcolor[rgb]{0.81,0.36,0.00}{\textbf{#1}}}
\newcommand{\OtherTok}[1]{\textcolor[rgb]{0.56,0.35,0.01}{#1}}
\newcommand{\PreprocessorTok}[1]{\textcolor[rgb]{0.56,0.35,0.01}{\textit{#1}}}
\newcommand{\RegionMarkerTok}[1]{#1}
\newcommand{\SpecialCharTok}[1]{\textcolor[rgb]{0.00,0.00,0.00}{#1}}
\newcommand{\SpecialStringTok}[1]{\textcolor[rgb]{0.31,0.60,0.02}{#1}}
\newcommand{\StringTok}[1]{\textcolor[rgb]{0.31,0.60,0.02}{#1}}
\newcommand{\VariableTok}[1]{\textcolor[rgb]{0.00,0.00,0.00}{#1}}
\newcommand{\VerbatimStringTok}[1]{\textcolor[rgb]{0.31,0.60,0.02}{#1}}
\newcommand{\WarningTok}[1]{\textcolor[rgb]{0.56,0.35,0.01}{\textbf{\textit{#1}}}}
\usepackage{longtable,booktabs}
% Correct order of tables after \paragraph or \subparagraph
\usepackage{etoolbox}
\makeatletter
\patchcmd\longtable{\par}{\if@noskipsec\mbox{}\fi\par}{}{}
\makeatother
% Allow footnotes in longtable head/foot
\IfFileExists{footnotehyper.sty}{\usepackage{footnotehyper}}{\usepackage{footnote}}
\makesavenoteenv{longtable}
\usepackage{graphicx}
\makeatletter
\def\maxwidth{\ifdim\Gin@nat@width>\linewidth\linewidth\else\Gin@nat@width\fi}
\def\maxheight{\ifdim\Gin@nat@height>\textheight\textheight\else\Gin@nat@height\fi}
\makeatother
% Scale images if necessary, so that they will not overflow the page
% margins by default, and it is still possible to overwrite the defaults
% using explicit options in \includegraphics[width, height, ...]{}
\setkeys{Gin}{width=\maxwidth,height=\maxheight,keepaspectratio}
% Set default figure placement to htbp
\makeatletter
\def\fps@figure{htbp}
\makeatother
\setlength{\emergencystretch}{3em} % prevent overfull lines
\providecommand{\tightlist}{%
  \setlength{\itemsep}{0pt}\setlength{\parskip}{0pt}}
\setcounter{secnumdepth}{5}

\title{SOLUCIONES: Ejercicios Tema 1 - Probabilidad.}
\author{Ricardo Alberich, Juan Gabriel Gomila y Arnau Mir}
\date{Curso de Probabilidad y Variables Aleatorias con R y Python}

\begin{document}
\maketitle

{
\hypersetup{linkcolor=blue}
\setcounter{tocdepth}{2}
\tableofcontents
}
\hypertarget{ejercicios-de-espacios-muestrales-y-sucesos}{%
\section{Ejercicios de Espacios muestrales y
sucesos}\label{ejercicios-de-espacios-muestrales-y-sucesos}}

\hypertarget{problema-1}{%
\subsection{Problema 1}\label{problema-1}}

Se seleccionan al azar tres cartas sin reposición de una baraja que
contiene 3 cartas rojas, 3 azules, 3 verdes y 3 negras. Especifica un
espacio muestral para este experimento y halla todos los sucesos
siguientes: * \(A\) = ``Todas las cartas seleccionadas son rojas'' *
\(B\) = ``Una carta es roja, 1 es verde y otra es azul'' * \(C\) =
``Salen tres cartas de colores diferentes''

\hypertarget{soluciuxf3n}{%
\subsubsection{Solución}\label{soluciuxf3n}}

Codificamos el espacio muestral por la inicial del color y el número de
carta de 1 a 3 por color
\[\Omega=\{R1, R2, R3, V1, V2, V3, A1, A2, A3  N1, N2, N3 \}.\]

Los sucesos que nos piden son:

\begin{itemize}
\tightlist
\item
  \(A=\{(R1, R2, R3),(R1, R3, R2),(R2, R1, R3),(R2, R3, R1),(R3, R1, R2), (R3, R2, R1)\}\),
\item
  \(B=\{(Ri,Vj,Ak),(Ri,Aj,Vk),(Vi,Rj,Ak),(Vi,Aj,Rk),(Ai,Vj,Rk),(Ai,Rj,Vk) | \mbox{ con } i,j,k\in\{1,2,3\}\}\).
  El cardinal de \(\left|B\right|=6\cdot 3\cdot 3\cdot 3\cdot 3= 162\)
\item
  Se selecciona tres colores de entre los 4 \({{4}\choose {3}}=4\)
  casos. Cada trío de colores tiene el mismo cardinal que \(B\) por lo
  tanto en total con son \(4\cdot 6\cdot 3^3=648\).
\end{itemize}

Ejercicio idead un algoritmo que escriba los objetos combinatorios en
cada caso.

\hypertarget{ejercicios-de-probabilidad}{%
\section{Ejercicios de Probabilidad}\label{ejercicios-de-probabilidad}}

\hypertarget{problema-1-1}{%
\subsection{Problema 1}\label{problema-1-1}}

Se lanzan al aire dos monedas iguales. Hallar la probabilidad de que
salgan dos caras iguales.

\hypertarget{soluciuxf3n-1}{%
\subsubsection{Solución}\label{soluciuxf3n-1}}

\(P(CC)=\frac{\left|\{CC\}\right|}{\left|\{CC,C+,+C,++\}\right|}=\frac{1}{4}.\)

\hypertarget{problema-2}{%
\subsection{Problema 2}\label{problema-2}}

Suponer que se ha trucado un dado de modo que la probabilidad de que
salga un número es proporcional al mismo.

\begin{itemize}
\tightlist
\item
  Hallar la probabilidad de los sucesos elementales, de que salga un
  número par y también de que salga un número impar.
\item
  Repetir el problema pero suponiendo que la probabilidad de que salga
  un determinado número es inversamente proporcional al mismo.
\end{itemize}

\hypertarget{soluciuxf3n-2}{%
\subsubsection{Solución}\label{soluciuxf3n-2}}

\textbf{PARTE I}

Sea \(\Omega=\{1,2,3,4,5,6\}\) y nos dicen que \(P(k)= \alpha\cdot k\).

La suma de todas las probabilidades debe ser 1 así que
\footnote{Utilizamos que
$\sum_{i=1}^n k=\frac{n\cdot (n+1)}{2}$ }
\(1=\sum_{k=1}^6 P(k)=\sum_{k=1}^6 \alpha \cdot k= \alpha \sum_{k=1}^6 k = \alpha \frac{6\cdot (6+1)}{2}=\alpha\cdot 21\).
Así tenemos que \(\alpha=\frac{1}{21}.\)

\(P(\mbox{salir par})= P(2)+P(4)+P(6)=\frac{2}{21}+\frac{4}{21}+\frac{6}{21}=\frac{12}{21}=\frac{4}{7}=0.5714286.\)

\(P(\mbox{salir impar})= 1- P(\mbox{salir par})=1-\frac{12}{21}=0.4285714.\)

\textbf{PARTE II}

Sea \(\Omega=\{1,2,3,4,5,6\}\) y nos dicen que
\(P(k)= \alpha\cdot \frac{1}{k}\).

La suma de todas las probabilidades debe ser 1 así que \[
1=\sum_{k=1}^6 P(k)=\sum_{k=1}^6 \alpha \cdot \frac{1}{k}=
\alpha \sum_{k=1}^6 k = \alpha \left(\frac{1}{1}+\frac{1}{2}+\frac{1}{3}+\frac{1}{4}+\frac{1}{5}+\frac{1}{6}\right)=\alpha\cdot 2.45,
\] Luego el valor buscado es \(\alpha=\frac{1}{2.45}=0.4081633\).

\(P(\mbox{salir par})= P(2)+P(4)+P(6)=0.4081633\left(\frac{1}{2}+\frac{1}{4}+\frac{1}{6}\right)=0.3741497.\)

\(P(\mbox{salir impar})= 0.4081633\left(\frac{1}{1}+\frac{1}{3}+\frac{1}{5}\right)=0.6258504.\)

Efectivamente
\(P(\mbox{salir impar})=1-P(\mbox{salir par})=1- 0.4081633=0.5918367.\)

\hypertarget{problema-3}{%
\subsection{Problema 3}\label{problema-3}}

En una prisión de 100 presos se seleccionan al azar dos personas para
ponerlas en libertad.

\begin{itemize}
\tightlist
\item
  ¿Cual es la probabilidad de que el más viejo de los presos sea uno de
  los elegidos?
\item
  ¿Y que salga elegida la pareja formada por el más viejo y el más
  joven?
\end{itemize}

\hypertarget{soluciuxf3n-3}{%
\subsubsection{Solución}\label{soluciuxf3n-3}}

Supongamos que solo hay un mas viejo y un más joven.

\begin{eqnarray*}
P(\mbox{De entre dos elegidos al azar esté el más viejo}) &=&\\
\frac{\mbox{Casos Favorables}}{\mbox{Casos posibles}}=
\frac{{{99}\choose {1}}}{{{100}\choose {2}}}=\frac{99}{\frac{100\cdot 99}{2}}&=&\\
\frac{2\cdot 99}{100\cdot 99}=
\frac{2}{100}=\frac{1}{50}.
\end{eqnarray*}

\begin{eqnarray*}
P(\mbox{De los dos elegidos al azar sean  el más viejo y el más joven})&=&\\
\frac{\mbox{Casos Favorables}}{\mbox{Casos posibles}}=
\frac{1}{{{100} \choose {2}}}=\frac{1}{\frac{100\cdot 99}{2}}&=&\\
\frac{2}{100\cdot 99}
=\frac{1}{4950}.
\end{eqnarray*}

\hypertarget{problema-4}{%
\subsection{Problema 4}\label{problema-4}}

Se apuntan tres corredores A, B y C a una carrera.

\begin{itemize}
\tightlist
\item
  ¿Cuál es la probabilidad de que A acabe antes que C si todos son igual
  de hábiles corriendo y no puede haber empates?
\item
  ¿Cuál es la probabilidad de que A acabe antes que B y C?
\end{itemize}

\hypertarget{soluciuxf3n-4}{%
\subsubsection{Solución}\label{soluciuxf3n-4}}

LoS casos posibles CP son \(3!=6\)

El corredor A puede acabar antes de C siendo primero o segundo en cada
caso \(2!\) casos (hacer un algoritmo que los escriba).

\[P(\mbox{A acabe antes de C})=\frac{CF}{CP}=\frac{2!+2!}{3!}=\frac{4}{6}=\frac{2}{3}.\]

\[P(\mbox{A acabe antes de B y C})=\frac{CF}{CP}=\frac{2!}{3!}=\frac{1}{3}.\]

\hypertarget{problema-5}{%
\subsection{Problema 5}\label{problema-5}}

En una sala se hallan \(n\) personas. ¿Cual es la probabilidad de que
haya al menos dos personas con el mismo mes de nacimiento? Dar el
resultado para los valores de \(n=3,4,5,6\).

\hypertarget{soluciuxf3n-5}{%
\subsubsection{Solución}\label{soluciuxf3n-5}}

\[P(\mbox{De n personas menos dos nazcan el mimo mes})=1-P(\mbox{De n personas tosdas nazca en distinto mes}).\]

Casos Posibles (CP) todas las maneras de escoger \(n\) meses de 12 esto
son \(VR_{12}^n=12^n.\).

Casos Favorables a que todas nazcan EN MESES DISTINTOS
\(V_{12}^n=\frac{12!}{(12-n)!}\)

\[
P(\mbox{De n personas menos dos nazcan el mimo mes})=
1-P(\mbox{De personas todas nazcan en distinto mes})=
1-\frac{\frac{12!}{(12-n)!}}{12^n}.
\]

Hagamos los calculamos con R para n=3,4,5,6

\begin{Shaded}
\begin{Highlighting}[]
\NormalTok{n=}\DecValTok{3}\OperatorTok{:}\DecValTok{6}
\KeywordTok{sapply}\NormalTok{(n,}\DataTypeTok{FUN=}\ControlFlowTok{function}\NormalTok{(n) }\DecValTok{1}\OperatorTok{{-}}\NormalTok{((}\KeywordTok{factorial}\NormalTok{(}\DecValTok{12}\NormalTok{)}\OperatorTok{/}\KeywordTok{factorial}\NormalTok{(}\DecValTok{12}\OperatorTok{{-}}\NormalTok{n)))}\OperatorTok{/}\NormalTok{(}\DecValTok{12}\OperatorTok{\^{}}\NormalTok{n))}
\end{Highlighting}
\end{Shaded}

\begin{verbatim}
## [1] 0.2361111 0.4270833 0.6180556 0.7771991
\end{verbatim}

\hypertarget{problema-6}{%
\subsection{Problema 6}\label{problema-6}}

Una urna contiene 4 bolas numeradas con los números 1, 2, 3 y 4,
respectivamente. Se sacan dos bolas sin reposición. Sea \(A\) el suceso
que la suma sea 5 y sea \(B_i\) el suceso que la primera bola extraída
tenga un \(i\),con \(i=1,2,3,4\). Hallar \(P(A|B_i),\  i=1,2,3,4\) y
\(P(B_i|A), i=1,2,3,4\).

\hypertarget{soluciuxf3n-6}{%
\subsubsection{Solución}\label{soluciuxf3n-6}}

Sin contamos el ``orden de extracción'' el espacio muestral son los
pares no ordenados
\[\Omega =\{(1,2),(2,1),(1,3),(3,1),(1,4),(4,1),(2,3),(3,2),(2,4),(4,2),(3,4),(4,3)\}.\]

Todos los pares son equiprobables y tienen por probabilidad
\(\frac{1}{6}.\)

El suceso de que la suma sea 5 es \(A=\{(1,4),(4,1),(3,2),(2,3)\}\). Los
sucesos que empiezan por \(i\) son \(B_i=\{(i,2),(i,3),(i,4)\}\) para
\(i=3,4,5,6.\) Es evidente que \(P(B_i)=\frac{3}{12}=\frac{1}{4}\)

\[
P(A|B_1)=\frac{P(A\cap B_1)}{P(B_1)}=\frac{P((1,4))}{\frac{3}{12}}=
\frac{\frac{1}{12}}{\frac{3}{12}}=\frac{1}{3}.
\]

\[
P(A|B_2)=\frac{P(A\cap B_2)}{P(B_2)}=\frac{P((2,3))}{\frac{4}{12}}=
\frac{\frac{1}{12}}{\frac{3}{12}}=\frac{1}{3}.
\]

\[
P(A|B_3)=\frac{P(A\cap B_3)}{P(B_3)}=\frac{P((3,2))}{\frac{3}{12}}=
\frac{\frac{1}{12}}{\frac{3}{12}}=\frac{1}{3}.
\]

\[P(A|B_4)=\frac{P(A\cap B_4)}{P(B_4)}=\frac{P((4,1))}{\frac{3}{12}}=
\frac{\frac{1}{12}}{\frac{3}{12}}=\frac{1}{3}.\]

Notemos que \(P(B_i)=\frac{3}{12}.\)

Ahora para hacer fácil la segunda parte ara \(i=1,2,3,4\) hacemos

\[
P(B_i/A)=\frac{P(A/B_i)\cdot P(B_i)}{P(A)}=\frac{\frac{1}{3}\cdot \frac{3}{12}}{\frac{4}{12}}=\frac{1}{4}.
\]

\hypertarget{problema-7}{%
\subsection{Problema 7}\label{problema-7}}

Se lanza al aire una moneda no trucada.

\begin{itemize}
\tightlist
\item
  ¿Cuál es la probabilidad que la cuarta vez salga cara, si sale cara en
  las tres primeras tiradas?
\item
  ¿Y si salen 2 caras en las 4 tiradas?
\end{itemize}

\hypertarget{soluciuxf3n-7}{%
\subsubsection{Solución}\label{soluciuxf3n-7}}

\[
P(\mbox{C en la cuarta}|CCC)=
\frac{P(\mbox{C en la cuarta}\cap \{CCC\})}{P(CCC)}=
\frac{P(CCCC)}{P(CCC)}=\frac{0.5^4}{0.5^3}=0.5
.\]

El suceso
\(\mbox{2 caras en cuatro tiradas}=\{CC++,C+C+,C++C,+C+C,++CC\}.\)

\begin{eqnarray*}
P(\mbox{C en la cuarta}|\{CC++,C+C+,C++C,+C+C,++CC\})&=&\\
\frac{P(\mbox{C en la cuarta}\cap \{CC++,C+C+,C++C,+C+C,++CC\})}{P(CCC)}&=&\\
\frac{P(\{C++C,+C+C,++CC\})}{P(CCC)}=\frac{3\cdot 0.5^4}{4\cdot 0.5^4}=\frac{3}{4}=0.75.
\end{eqnarray*}

\hypertarget{problema-8}{%
\subsection{Problema 8}\label{problema-8}}

La urna 1 contiene 2 bolas rojas y 4 de azules. La urna 2 contiene 10
bolas rojas y 2 de azules. Si escogemos al azar una urna y sacamos una
bola,

\begin{itemize}
\tightlist
\item
  ¿Cuál es la probabilidad que la bola seleccionada sea azul?
\item
  ¿Y que sea roja?
\end{itemize}

\hypertarget{soluciuxf3n-8}{%
\subsubsection{Solución}\label{soluciuxf3n-8}}

Sea \(U_i\) el suceso hemos es cogido la urna \(i\) para \(i=1,2\), y
\(A\) el suceso la bola extraída es azul y \(R\) roja.

Sabemos que \(P(A/U_1)=\frac{4}{6}\), \(P(A/U_2)=\frac{2}{12}\),
\(P(R/U_1)=\frac{2}{6}\) y \(P(R/U_2)=\frac{10}{12}.\)

Nos piden en primer lugar \(P(A)\) sin saber de qué urna se ha extraído.
Utilizaremos el teorema de la probabilidad total

\[
P(A)=P(A|U_1)\cdot P(U_1)+P(A|U_2)\cdot P(U_2)=
\frac{4}{6}\cdot \frac{1}{2}+\frac{2}{12}\cdot \frac{1}{2}
=\frac{4}{12}+\frac{1}{12}=\frac{5}{12}.
\]

Por otra parte

\(P(R)=1-P(A)=1-\frac{5}{12}=\frac{7}{12}.\)

\hypertarget{problema-9}{%
\subsection{Problema 9}\label{problema-9}}

Supongamos que la ciencia médica ha desarrollado una prueba para el
diagnóstico de cáncer que tiene un 95\% de exactitud, tanto en los que
tienen cáncer como en los que no. Si el 5 por mil de la población
realmente tiene cáncer, encontrar la probabilidad que un determinado
individuo tenga cáncer, si la prueba ha dado positiva.

\hypertarget{soluciuxf3n-9}{%
\subsubsection{Solución}\label{soluciuxf3n-9}}

Sea \$C4 es el suceso tener cáncer, y T el sucesos el diagnóstico da
positivo en cáncer.

Nos dicen que \(P(T|C)=P(T^c|C^c)=0.95\) y que \(P(C)=0.005\) por lo
tanto \(P(T|C^c)=1-P(T^c|C^2)=0.05\) y \(P(C^c)=1-P(C)=0.005\).

Nos piden \[
P(C|T)=\frac{P(T|C)\cdot P(C)}{P(T)}= 
\frac{P(T|C)\cdot P(C)}{P(T|C)\cdot P(C)+P(T|C^c)\cdot P(C^c)} =
\frac{0.95\cdot 0.005}{0.95\cdot 0.005+(1-0.95)\cdot (1-0.005)}=
0.087156. 
\]

\hypertarget{problema-10}{%
\subsection{Problema 10}\label{problema-10}}

Se lanzan una sola vez dos dados. Si la suma de los dos dados es como
mínimo 7, ¿cuál es la probabilidad que la suma sea igual a \(i\), para
\(i=7,8,9,10,11,12\)?

\hypertarget{soluciuxf3n-10}{%
\subsubsection{Solución}\label{soluciuxf3n-10}}

Estos son los resultados de la suma sin la condición

\begin{Shaded}
\begin{Highlighting}[]
\NormalTok{dado1=}\KeywordTok{rep}\NormalTok{(}\DecValTok{1}\OperatorTok{:}\DecValTok{6}\NormalTok{,}\DataTypeTok{times=}\DecValTok{6}\NormalTok{)}\CommentTok{\# dado 1}
\NormalTok{dado2=}\KeywordTok{rep}\NormalTok{(}\DecValTok{1}\OperatorTok{:}\DecValTok{6}\NormalTok{,}\DataTypeTok{each=}\DecValTok{6}\NormalTok{) }\CommentTok{\#dado 2}
\NormalTok{resultados=}\KeywordTok{data.frame}\NormalTok{(dado1,dado2,}\DataTypeTok{suma=}\NormalTok{dado1}\OperatorTok{+}\NormalTok{dado2)}
\NormalTok{knitr}\OperatorTok{::}\KeywordTok{kable}\NormalTok{(resultados)}
\end{Highlighting}
\end{Shaded}

\begin{longtable}[]{@{}rrr@{}}
\toprule
dado1 & dado2 & suma\tabularnewline
\midrule
\endhead
1 & 1 & 2\tabularnewline
2 & 1 & 3\tabularnewline
3 & 1 & 4\tabularnewline
4 & 1 & 5\tabularnewline
5 & 1 & 6\tabularnewline
6 & 1 & 7\tabularnewline
1 & 2 & 3\tabularnewline
2 & 2 & 4\tabularnewline
3 & 2 & 5\tabularnewline
4 & 2 & 6\tabularnewline
5 & 2 & 7\tabularnewline
6 & 2 & 8\tabularnewline
1 & 3 & 4\tabularnewline
2 & 3 & 5\tabularnewline
3 & 3 & 6\tabularnewline
4 & 3 & 7\tabularnewline
5 & 3 & 8\tabularnewline
6 & 3 & 9\tabularnewline
1 & 4 & 5\tabularnewline
2 & 4 & 6\tabularnewline
3 & 4 & 7\tabularnewline
4 & 4 & 8\tabularnewline
5 & 4 & 9\tabularnewline
6 & 4 & 10\tabularnewline
1 & 5 & 6\tabularnewline
2 & 5 & 7\tabularnewline
3 & 5 & 8\tabularnewline
4 & 5 & 9\tabularnewline
5 & 5 & 10\tabularnewline
6 & 5 & 11\tabularnewline
1 & 6 & 7\tabularnewline
2 & 6 & 8\tabularnewline
3 & 6 & 9\tabularnewline
4 & 6 & 10\tabularnewline
5 & 6 & 11\tabularnewline
6 & 6 & 12\tabularnewline
\bottomrule
\end{longtable}

\begin{Shaded}
\begin{Highlighting}[]
\KeywordTok{table}\NormalTok{(resultados}\OperatorTok{$}\NormalTok{suma)}
\end{Highlighting}
\end{Shaded}

\begin{verbatim}
## 
##  2  3  4  5  6  7  8  9 10 11 12 
##  1  2  3  4  5  6  5  4  3  2  1
\end{verbatim}

Si condicionamos a que la suma es mayor o igual a 7

\begin{Shaded}
\begin{Highlighting}[]
\KeywordTok{table}\NormalTok{(resultados}\OperatorTok{$}\NormalTok{suma[resultados}\OperatorTok{$}\NormalTok{suma}\OperatorTok{\textgreater{}=}\DecValTok{7}\NormalTok{])}
\end{Highlighting}
\end{Shaded}

\begin{verbatim}
## 
##  7  8  9 10 11 12 
##  6  5  4  3  2  1
\end{verbatim}

La probabilidades pedidas son

\begin{Shaded}
\begin{Highlighting}[]
\KeywordTok{prop.table}\NormalTok{(}\KeywordTok{table}\NormalTok{(resultados}\OperatorTok{$}\NormalTok{suma[resultados}\OperatorTok{$}\NormalTok{suma}\OperatorTok{\textgreater{}=}\DecValTok{7}\NormalTok{]))}
\end{Highlighting}
\end{Shaded}

\begin{verbatim}
## 
##          7          8          9         10         11         12 
## 0.28571429 0.23809524 0.19047619 0.14285714 0.09523810 0.04761905
\end{verbatim}

Como ejercicio formalizar con teoría de la probabilidad básica alguno de
los resultados y comprobad que es correcto.

\hypertarget{problema-11.}{%
\subsection{Problema 11.}\label{problema-11.}}

Se sabe que \({2 \over 3}\) de los internos de una cierta prisión son
menores de 25 años. También se sabe que \({3\over 5}\) son hombres y que
\({5\over 8}\) de los internos son mujeres o mayores de 25 años. ¿Cuál
es la probabilidad de que un prisionero escogido al azar sea mujer y
menor de 25 años?

\hypertarget{soluciuxf3n-11}{%
\subsubsection{Solución}\label{soluciuxf3n-11}}

Denotemos por \(V=\) " ser menor de 25 años\$ por \(M=\) ser mujer y
\(M^c=H\) ser hombre . Los datos que nos dan son \(P(V)=\frac{2}{3}\),
\(P(H)= \frac{3}{5}\) y \(P(M\cup V^c )=\frac{5}{8}.\)

Así que
\(\frac{5}{8}=P(M\cup V^c)=P(H^c\cup V^c)=P((H\cap V)^c)=1-P(H\cap V)\)
y por lo tanto \(P(H\cap V)=1- \frac{5}{8}=\frac{3}{8}.\)

Nos piden
\(P(M\cap V)=P(M)+P(V)-P(M\cup V)=1-P(H)+P(V)-P(M\cup V)=1-\frac{3}{5}+\frac{2}{3}-P(M\cup V).\)

Ahora
\(\frac{5}{8}=P(M\cup V^c)=P(M)-P(V)-P(M\cap V^c)=1-\frac{3}{5}+\frac{2}{3}-P(M\cap V^C).\)
luego tenemos que \$P(M\cap V\^{}c)=
\frac{5}{8}-1+\frac{3}{5}-\frac{2}{3}=

\hypertarget{problema-12.}{%
\subsection{Problema 12.}\label{problema-12.}}

Consideremos una hucha con \(2n\) bolas numeradas del \(1\) al \(2n\).
Sacamos \(2\) bolas de la urna sin reposición. Sabiendo que la segunda
bola es par, ¿cuál es la probabilidad de que la primera bola sea impar?

\hypertarget{soluciuxf3n-12}{%
\subsubsection{Solución}\label{soluciuxf3n-12}}

\hypertarget{problema-13.}{%
\subsection{Problema 13.}\label{problema-13.}}

Consideramos el siguiente experimento aleatorio: sacamos \(5\) números
al azar sin reposición a partir de los números naturales
\(1,2,\dots,20\). Encontrad la probabilidad \(p\) de que haya
exactamente dos números tales que sean múltiplos de \(3\)

\hypertarget{soluciuxf3n-13}{%
\subsubsection{Solución}\label{soluciuxf3n-13}}

\hypertarget{problema-14.}{%
\subsection{Problema 14.}\label{problema-14.}}

En una hucha hay \(10\) bolas, numeradas del \(1\) al \(10\). Las \(4\)
primeras bolas, o sea, las bolas \(1,2,3,4\) son blancas. Las bolas
\(5,6\) son negras y las bolas restantes son rojas. Sacamos dos bolas
sin reposición. Sabiendo que la segunda bola es de color negro,
encuentra la probabilidad \(p\) de que la primera bola sea blanca.

\hypertarget{soluciuxf3n-14}{%
\subsubsection{Solución}\label{soluciuxf3n-14}}

\hypertarget{problema-15.}{%
\subsection{Problema 15.}\label{problema-15.}}

Lanzamos un dado no trucado 3 veces. Encontrad la probabilidad \(p\) de
que la suma de las 3 caras sea \(10\).

\hypertarget{soluciuxf3n-15}{%
\subsubsection{Solución}\label{soluciuxf3n-15}}

\hypertarget{problema-16.}{%
\subsection{Problema 16.}\label{problema-16.}}

Tenemos 4 cartas numeradas del 1 al 4 que están giradas boca abajo sobre
una mesa. Una persona, supuestamente adivina, irá adivinando los valores
de las 4 cartas una a una. Suponiendo que es un farsante y que lo que
hace es decir los 4 números al azar, ¿cuál es la probabilidad de que
acierte como mínimo 1? (Obviamente, no repite ningún número)

\hypertarget{soluciuxf3n-16}{%
\subsubsection{Solución}\label{soluciuxf3n-16}}

\hypertarget{problema-17.}{%
\subsection{Problema 17.}\label{problema-17.}}

Una forma de aumentar la fiabilidad de un sistema es mediante la
introducción de una copia de los componentes en una configuración
paralela. Supongamos que la NASA quiere una probabilidad no menor que
0.99999 de que el transbordador espacial entre en órbita alrededor de la
Tierra con éxito. ¿Cuántos motores se deben configurar en paralelo para
que se consiga dicha fiabilidad si se sabe que la probabilidad de que un
motor funcione adecuadamente es 0.95? Supongamos que los motores
funcionan de manera independiente los unos con los otros.

\hypertarget{ejercicios-de-independencia-de-sucesos}{%
\section{Ejercicios de Independencia de
sucesos}\label{ejercicios-de-independencia-de-sucesos}}

\hypertarget{problema-1.}{%
\subsection{Problema 1.}\label{problema-1.}}

Una moneda no trucada se lanza al aire 2 veces Consideremos los
siguientes sucesos:

\begin{itemize}
\item
  A: Sale una cara en la primera tirada.
\item
  B: Sale una cara en la segunda tirada.

  ¿Son los sucesos A y B independientes?
\end{itemize}

\hypertarget{soluciuxf3n-17}{%
\subsubsection{Solución}\label{soluciuxf3n-17}}

\hypertarget{problema-2.}{%
\subsection{Problema 2.}\label{problema-2.}}

Una urna contiene 4 bolas numeradas con los números 1, 2, 3 y 4,
respectivamente. Se extraen dos bolas sin reposición. Sea A el suceso
que la primera bola extraída tenga un 1 marcado y sea B el suceso que la
segunda bola extraída tenga un 1 marcado.

\begin{itemize}
\tightlist
\item
  ¿Se puede decir que A y B son independientes?
\item
  ¿Y si el experimento fuera sin reposición?
\end{itemize}

\hypertarget{soluciuxf3n-18}{%
\subsubsection{Solución}\label{soluciuxf3n-18}}

\hypertarget{problema-3.}{%
\subsection{Problema 3.}\label{problema-3.}}

Sea \(\Omega\) un espacio muestral y \(A,B,C\) tres sucesos. Probad que

\begin{itemize}
\tightlist
\item
  Si \(A\) y \(B\) son independientes, también lo son \(A\) y \(B^c\)
\item
  Si \(A,B,C\) son independientes, también lo son \(A,B\) y \(C^c\)
\item
  ¿Es cierto que si \(A,B,C\) son independientes, también lo son
  \(A,B^c\) y \(C^c\)? ¿Y \(A^c, B^c\) y \(C^c\)? En caso de que la
  respuesta sea negativa, dad contra ejemplos donde la propiedad falle.
\end{itemize}

\hypertarget{soluciuxf3n-19}{%
\subsubsection{Solución}\label{soluciuxf3n-19}}

\hypertarget{problema-4.}{%
\subsection{Problema 4.}\label{problema-4.}}

Dos empresas \(A\) y \(B\) fabrican el mismo producto. La empresa \(A\)
tiene un \(2\%\) de productos defectuosos mientras que la empresa \(B\)
tiene un \(1\%\). Un cliente recibe un pedido de una de las empresas (no
sabe cuál) y comprueba que la primera pieza funciona. Si suponemos que
el estado de las piezas de cada empresa es independiente, ¿cuál es la
probabilidad de que la segunda pieza que pruebe sea buena? Comprobad que
el estado de las dos piezas no es independiente, pero en cambio es
condicionalmente independiente dada la empresa que las fabrica.

\hypertarget{soluciuxf3n-20}{%
\subsubsection{Solución}\label{soluciuxf3n-20}}

\hypertarget{problema-5.}{%
\subsection{Problema 5.}\label{problema-5.}}

Encuentra un ejemplo de tres sucesos \(A,B,C\) tales que \(A\) y \(B\)
sean independientes, pero en cambio no sean condicionalmente
independientes dado \(C\).

\end{document}
